%% This LaTeX-file was created by <karsten> Sun Aug 16 17:04:20 1998
%% LyX 0.12 (C) 1995-1998 by Matthias Ettrich and the LyX Team

%% Do not edit this file unless you know what you are doing.
\documentclass[12pt,english]{article}
\usepackage[T1]{fontenc}
\usepackage{palatino}
\usepackage{a4}
\usepackage{babel}

\makeatletter


%%%%%%%%%%%%%%%%%%%%%%%%%%%%%% LyX specific LaTeX commands.
\newcommand{\LyX}{L\kern-.1667em\lower.25em\hbox{Y}\kern-.125emX\spacefactor1000}

\makeatother

\begin{document}


\title{\textsl{M} - Introduction}


\author{Karsten Ball�der\\ (\texttt{Ballueder@usa.net,~http://Ballueder.home.ml.org})}

\maketitle
\begin{abstract}
This is a first attempt at some form of documentation for M. At present there
is no proper help system, so we haven't gotten around to write any real documentation
yet. M is beginning to get usable and so this should help you to set it up and
start using it. It also contains some information on what we plan for the future.

Some more up to date information can be found in the README files.
\end{abstract}
\tableofcontents


\section{Introduction }

M is intended to be a powerful mail and news client. The following points were
significant in the development of it 

\begin{itemize}
\item it should have an easy to use graphical user interface 
\item it should be as powerful and extensible as possible, being able to replace awkward
tools such as procmail, thus it must have a built in scripting language 
\item it should have full support for a wide range of standards, including full and
intuitive MIME support 
\item it should interact with other programs intuitively, sending MIME mails should
be as easy as dropping files from a file manager in the message window
\item it should run on a wide variety of systems 
\end{itemize}

\subsection{Current Status}

As of the time of writing this document, M is under active development. A first
public alpha release has been made.


\subsection{Features already implemented}

Many of the features planned are already available and new ones are added all
the time. This list will change constantly.

\begin{itemize}
\item Cross-platform. M compiles on a variety of Unix systems and on Microsoft Windows.
Use one mail client, no matter what system you use.
\item Based on the c-client library from the University of Washington, therefore full
access to a wide range of protocols and file formats, including SMTP, MAP, POP3,
NNTP and several mailbox formats. This is the same library as used by PINE,
so it is well tested and reliable.
\item Wide (extreme?) user configurability. Whatever makes sense to override or change,
can be changed by the user. Configuration supports several configuration files
on Unix, with special administrator support for making entries immutable, and
the registry on Windows.
\item Scriptable and extendable. M includes an embedded Python interpreter with full
access to its object hierarchy. Write object-oriented scripts to extend and
control M.
\item Easy MIME support. Text and other content can be freely mixed and different
filetypes are represented by icons.
\item Inline displaying of images.
\item Clickable URLs.
\item XFace support.
\item Multiple mail folders.
\item Powerful address database and contact manager with automated collection of data
from mails.
\item Printing of nicely formatted messages.
\item Full internationalisation support, M speaks multiple languages.
\end{itemize}

\subsection{Features still missing}

This is a list of features on our TODO list that we are currently working on.

\begin{itemize}
\item Full Drag and Drop interaction with filemanagers of Windows and Gnome (will
be added real soon, easy).
\item Easy to use filtering system for mails.
\item Support for V-cards.
\item Nested mail folder hierarchy.
\item Spam-Ex spam fighting/auto-complaint function.
\item Richt-text editing and HTML mail support
\item Support for PGP and GNU Privacy Guard to encrypt mails.
\item Threading of messages and proper usenet news support.
\item Compression of mail folders.
\item Delay-Folder to keep mails and re-present them at a later date.
\item Context sensitive help system (HTML based).
\item Translations to German, French and Italian.
\item Wide character (Unicode) support and other character sets.
\item Import, export and synchronisation with other programs' address databases.
\item Voice mail.
\item More Python support through wxPython.
\item Support for Drag and Drop interaction with KDE, once that wxQt is available.
\item CORBA support, possible cooperation with PINN project.
\item Most of the GUI issues will disappear as wxGTK evolves, which is still in ALPHA
stage und under heavy development itself.
\end{itemize}

\subsection{Help Needed}

As you can see, we have big plans for M. To achieve all this, we need some help.
Areas where we would use some help are

\begin{itemize}
\item The Python support could be improved (We are absolute beginners in Python, so
help is very welcome.). 
\item Also, if you want to add support for further mail protocols, please get in touch
with us. 
\item The wxQt project, a port of wxWindows to the Qt toolkit, will also be happy
for any help. We are not directly involved in this, but being involved with
wxWindows, we are happy to support that port.
\item If you have access to other systems apart from Linux/Solaris/Windows, you are
very welcome to help us port M to those platforms, or to other hardware than
Intel.
\item If you have anything that you would like to change, improve or add to M, please
get in touch with us. We want M to be the best and if you miss a feature, we
can always add it.
\end{itemize}

\section{Compilation notes}

These compilation notes are probably a bit outdated. The best start is to use
the \texttt{-{}-help} option of configure to see which options it supports.


\subsection{Operating systems specific}


\subsubsection{Linux}

If compiling with a non-default compiler like \texttt{egcs}, make sure that
\texttt{/usr/include} is not in the include path, neither should \texttt{/usr/lib}
be explicitly listed. M has been compiled with \texttt{egcs} and \texttt{gcc-2.8.x}
on both, \texttt{libc5} and \texttt{glibc2} systems.


\subsubsection{Solaris/SunOS}

M has been successfully compiled with \texttt{gcc-2.8.0} on Solaris. Currently
it does not compile with the standard \texttt{C++} compiler.


\subsubsection{Microsoft Windows }

M can be compiled under Windows, using wxWindows Version 2.0 and Microsoft Visual
C++.


\subsection{Other issues/libraries}


\subsubsection{C-client library}

A copy of the~\texttt{c-client~library} is required and is included with the
M sources. It is available separately from\\~\texttt{ftp://ftp.cac.washington.edu/imap/imap.tar.Z}.
Before compiling it with M, you need to patch it. 

The following information only applies if you use a separate \texttt{c-client}
library source:

Unpack the archive in the main M directory, then change into the IMAP directory
and try a first \texttt{make}, i.e. a \texttt{make~linux} or \texttt{make~gso}.
This will create a source directory \texttt{c-client} with lots of links to
other source files in it. 

Then install the \texttt{rfc822.c.patch} on c-client's \texttt{rfc822.c} file
from the \texttt{extra/patches} subdirectory of M and run the \texttt{c-client++}
script from the \texttt{extra/scripts} subdirectory in the c-client source directory,
which will rename variables in the c-client code to make it C++ compliant. If
there is any problem, it helps to edit the \texttt{CCTYPE} or \texttt{CFLAGS}
files. After creation of the library \texttt{c-client.a}, all object files can
be deleted.


\subsubsection{Python }

\texttt{configure} looks for Python in \texttt{/usr/local/src/Python-1.5}. If
your Python is installed in a different location, change the variable \texttt{PYTHON\_PATH}
at the beginning of \texttt{configure}. 


\subsubsection{XFaces }

If you have the \texttt{compface} library and header file installed, it will
be used to support XFaces. To install it, unpack it under the main M directory
and apply the patch \texttt{compface.patch} from the \texttt{extra/patches}
directory to its sources. Compile it and link its header and library to the
\texttt{extra/include} and \texttt{extra/lib} directories.


\section{Installation}


\subsection{Configuration and Compilation}

Follow these steps: 

\begin{enumerate}
\item Edit \texttt{configure} so it will find your installation of Python. If you
do not have it, just skip this step.
\item Run \texttt{configure} to create the \texttt{include/config.h} and \texttt{makeopts}
file. It may be required to edit \texttt{makeopts} by hand.
\item Run \texttt{make~dep} and \texttt{cd~src} and \texttt{make} to compile and link
M. Compiling some of the source files will take an enormous amount of memory,
so make sure you have enough virtual memory. 
\end{enumerate}

\section{Configuration and Testing}


\subsection{Configuration settings}

Under Unix all configuration settings are stored in \texttt{\~{}/.M/config}
under Windows in the registry. To get an overview over all possible configuration
options and their default values, set the value \texttt{RecordDefaults=1}. Under
Unix, do this by creating a new \texttt{\~{}/.M/config} file containig the lines

\verb|[M]|

\verb|RecordDefaults=1|

After running M, this file will then contain all default settings. Most of them
are easily understood. Otherwise, the file \texttt{include/Mdefaults.h} contains
them all with some short comments.


\subsection{Reading mail/news}

To read news, you need to open a folder. The default incoming mail folder has
the name \texttt{INBOX}. Any other name will be interpreted as a filename relative
to the folder directory. 


\subsection{Writing mail/news}

Before being able to send mail, you need to configure the \texttt{MailHost}
setting to tell it where to send the mail. 


\section{Configuration Settings}

Configuration settings are stored in configuration files under Unix and in the
Registry under Windows. There are two types of configuration entries, global
settings and profile settings. Global settings are all top-level configuration
entries.

To get a list of all default entries under Unix, create an empty configuration
file \texttt{\$HOME/.M/config} with the content

\texttt{{[}M{]}}

\texttt{RecordDefaults=1}


\subsection{Profile Settings}

M uses a system of inheriting configurations for most of its settings. Each
profile is represented by one entry in the configuration hierarchy. For example,
the settings for a standard message window (MessageView) are stored in an entry
\texttt{{[}M/Profiles/MessageView{]}}.

As the message window is opened from within a mail folder, it also inherits
the profile from the mail folder, which might be \texttt{{[}M/Profiles/INBOX{]}}or
\texttt{{[}M/Profiles/ReallyInterestingMail{]}}, depending on which folder you
are using at the moment. The mail folder setup itself might inherit from some
other profile and eventually all profiles inherit from \texttt{{[}M/Profiles{]}}.


\section{Scripting and Python integration}


\subsection{Introduction}

M uses Python as an embedded scripting language. A large number of user definable
callback functions are available. Scripts have access most objects living in
M.


\subsection{Initialisation }

At startup, M will load a file called \texttt{Minit.py} and call the \texttt{Minit()}
function defined in there, without any arguments.


\subsection{Callback Functions (Hooks)}

There are a large number of callbacks available which will be called from different
places withing M. These are documented in \texttt{Mcallbacks.h}. All of these
callbacks are called with at least two arguments:

\begin{enumerate}
\item The name of the hook for which the function got called, e.g. \texttt{FolderOpenHook}
\item A pointer to the object from which it was called. E.g. for \texttt{FolderOpenHook},
this would be a pointer to a \texttt{MailFolder} object. This object does not
carry a useable type with it and needs to be converted in the callback, e.g.
if the argument is called \texttt{arg} and the object is a \texttt{MailFolder},
the object must either be used as \texttt{MailFolder.MailFolder(arg)} or be
converted as \texttt{mf~=~MailFolder.MailFolder(arg)}. 
\item Some callbacks have a third argument. This is either a single value or a tuple
holding several values.
\end{enumerate}

\subsection{Namespaces}

To avoid repeatedly typing in the name of the module (\texttt{MailFolder} in
this case), it can be imported into the global namespace with ``\texttt{from\~{}MailFolder\~{}import\~{}{*}}''.
By default modules are not imported into the global namespace and must be explicitly
named.


\subsection{List of Callbacks}

\vspace{0.3cm}
{\centering \begin{tabular}{|l|c|l|l|l|}
\hline 
Callback Name&
Object Type&
Additional Arguments/Types&
Return Value&
Documentaion\\
\hline 
\hline 
FolderOpenHook&
MailFolder&
&
void&
Called after a folder has been opened.\\
\hline 
FolderUpdateHook&
MailFolder&
&
void&
Called after a folder has been updated.\\
\hline 
FolderSetMessageFlag&
MailFolder&
(long) index of message&
1 if changing flags is ok,0 otherwise&
Called before changing flags for a mesage.\\
&
&
(string)name of flag&
&
\\
\hline 
FolderClearMessageFlag&
MailFolder&
(long) index of message&
1 if changing flags is ok,0 otherwise&
Called before changing flags for a mesage.\\
&
&
(string) name of flag&
&
\\
\hline 
FolderExpungeHook&
MailFolder&
&
1 to expunge, 0 to abort&
Called before expunging messages.\\
\hline 
\end{tabular}\par}
\vspace{0.3cm}


\section{Further Information}

\begin{itemize}
\item You can download the latest version of M from \texttt{http://Ballueder.home.ml.org/M/} 
\item You can also get up-to-date information on M from the M WWW Page:~\texttt{http://Ballueder.home.ml.org/M/}
\item wxWindows is available from \texttt{http://web.ukonline.co.uk/julian.smart/}
\item The GTK port of wxWindows, wxGTK, is available from: http://www.freiburg.linux.de/\~{}wxxt/
\end{itemize}

\section{FAQ}

\end{document}
